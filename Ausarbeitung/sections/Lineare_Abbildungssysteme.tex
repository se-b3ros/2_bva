\section{Lineare Abbildungssysteme}

\subsection{Wie funktioniert die Lochkamera (Kamera-Obscura) im Detail?}
Das grundlegende Funktionsprinzip der Kamera-Obscura basiert darauf, dass durch ein kleines Loch auf der einen Seite eines 
lichtdichten Raums die vom Motiv reflektierten Strahlen auf die gegenüberliegende Seite fallen und dort das Objekt seitenverkehrt 
und kopfstehend abbilden.

\includeImgNoUrl{H}{width=0.5\linewidth}{./img/Lineare_Abbildungssysteme/1.png}{Konzept der Kamera Obscura}{fig:linsys-1}{\centering}

\textbox{
  Die Größe der Lochblende ist ein Kompromiss zwischen Schärfe und Helligkeit.\\ \textbf{Je größer das Loch $\Rightarrow$ desto heller 
  und ebenso weniger scharf das Bild}
}

\noindent

\subsubsubsection{Kamera-Koordinaten vs. Bild-Koordinaten}

\begin{table}[H]
  \begin{tabularx}{\linewidth}{| X | X |}
    \hline
    \rowcolor{Gray} \textbf{Kamera-Koordinaten} & \textbf{Bild-Koordinaten} \\\hline

    3D (x, y, z) & 2D ($x_b$, $y_b$)\\\hline
    Beschreibt die Position in der Realität relativ zur Kameraposition & Ist die Projektion der 3D Wirklichkeit auf die 2D-Bildebene\\\hline
  \end{tabularx}
\end{table}

\includeImgNoUrl{H}{width=\linewidth}{./img/Lineare_Abbildungssysteme/2.png}{Geometrie Camera-Obscura}{fig:linsys-2}{\centering}

\noindent
Wenn $z < b$ (z ist der Abstand von der Lochblende zum Objekt und b ist der Abstand von der Lochblende zur Bildfläche(Rückwand)), dann
wird das Objekt auf der Bildfläche \textbf{vergrößert} dargestellt. Ist jedoch $z > b$ wird das Objekt \textbf{verkleinert} dargestellt.

\subsection{Welche Auswirkungen hat die Lochblende bei der Kamera-Obscura?}

\subsection{Was passiert, wenn man das Loch bei der Kamera-Obscura vergrößert / verkleinert?}

\subsection{Wovon hängt die schärfe des Bildes bei der Kamera-Obscura ab?}

\subsection{Wovon hängt die Helligkeit des Bildes bei der Kamera-Obscura ab?}

\subsection{In welchem Zusammenhang stehen Helligkeit und Schärfe bei der Kamera-Obscura im Bezug auf die Fokuslänge?}

\subsection{Was sind intrinsische und extrinische Ebenen, Parameter?}

\subsection{Welche intrinischen, extrinsischen Parameter gibt es?}

\subsection{Welche Arten der Verzerrung gibt es bei der Kamera-Kalibrierung?}

\subsection{Wie werden Fehler bei der Kamera-Kalibrierung minimiert?}

\subsection{Was passiert auf dem Bild auf Folie 37 (Kamera Kalibrierung, x und w auf Linie bringen)?}

\subsection{Wie funktioniert Kamera Kalibrierung? (Allgemein und Übung)}

\subsection{Nennen sie die Matrizen für die intrinischen und extrinsischen Parameter?}

\subsection{Was sind Vorteile der intrinischen im Vergleich zu den extrinischen Parametern? (Kommutativität)}

\subsection{Wie können die Kamera Parameter angewendet/umgerechnet werden?}

\subsection{Wie funktionieren 2D Koordinatensysteme?}

\subsection{Was ist der Einheitsvektor in einem 2D-Koordinatensystem?}

\subsection{Was ist mit "Projektion" gemeint? (2D-Koordinatensystem)}

\subsection{Welche Matrix-Operationen können im 2D-Koordinatensystem angewendet werden? (Beispiele)}


