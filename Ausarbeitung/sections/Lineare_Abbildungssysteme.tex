\section{Lineare Abbildungssysteme}

\subsection{Wie funktioniert die Lochkamera (Kamera-Obscura) im Detail?}
Das grundlegende Funktionsprinzip der Kamera-Obscura basiert darauf, dass durch ein kleines Loch auf der einen Seite eines 
lichtdichten Raums die vom Motiv reflektierten Strahlen auf die gegenüberliegende Seite fallen und dort das Objekt seitenverkehrt 
und kopfstehend abbilden.

\includeImgNoUrl{H}{width=0.5\linewidth}{./img/Lineare_Abbildungssysteme/1.png}{Konzept der Kamera Obscura}{fig:linsys-1}{\centering}

\textbox{
  Die Größe der Lochblende ist ein Kompromiss zwischen Schärfe und Helligkeit.\\ \textbf{Je größer das Loch $\Rightarrow$ desto heller 
  und ebenso weniger scharf das Bild}
}

\noindent

\subsubsubsection{Kamera-Koordinaten vs. Bild-Koordinaten}

\begin{table}[H]
  \begin{tabularx}{\linewidth}{| X | X |}
    \hline
    \rowcolor{Gray} \textbf{Kamera-Koordinaten} & \textbf{Bild-Koordinaten} \\\hline

    3D (x, y, z) & 2D ($x_b$, $y_b$)\\\hline
    Beschreibt die Position in der Realität relativ zur Kameraposition & Ist die Projektion der 3D Wirklichkeit auf die 2D-Bildebene\\\hline
  \end{tabularx}
\end{table}

\includeImgNoUrl{H}{width=\linewidth}{./img/Lineare_Abbildungssysteme/2.png}{Geometrie Camera-Obscura}{fig:linsys-2}{\centering}

\noindent
Wenn $z < b$ (z ist der Abstand von der Lochblende zum Objekt und b ist der Abstand von der Lochblende zur Bildfläche(Rückwand)), dann
wird das Objekt auf der Bildfläche \textbf{vergrößert} dargestellt. Ist jedoch $z > b$ wird das Objekt \textbf{verkleinert} dargestellt.

\subsection{Welche Auswirkungen hat die Lochblende bei der Kamera-Obscura?}
Die Lochblende bei der Kamera-Obscura ist die einzige Öffnung im geschlossenen Raum, durch die Licht in den Raum kommt. Dabei ist die 
Größe der Öffnung maßgeblich für die Helligkeit und auch für die Schärfe des 2D Bildes auf der Rückseite des Raums.

\textbox{
  \textbf{Je Größer die Öffnung, desto heller, aber gleichzeitig weniger scharf das 2D Bild.}
}

\noindent
Zudem ist das Verhältnis der Abstände zwischen Objekt und Lochlende zu Bildfläche und Lochblende entscheidend für den "Zoom".\\ 

\textbox{
  Ist der Abstand zwischen Objekt und Lochblende \textbf{kleiner} als der Abstand zwischen Lochblende und Bildfläche, so wird das 
  Objekt auf der Bildfläche \textbf{vergrößert}, andernfalls \textbf{verkleinert}
}

\subsection{Was passiert, wenn man das Loch bei der Kamera-Obscura vergrößert / verkleinert?}
Wird das Loch vergrößert, so wird das Bild auf der Bildfläche heller, aber auch unschärfer. Andernfalls wird das Bild dunkler 
(es kommt weniger Licht in den Raum), aber das Bild erscheint schärfer.

\subsection{Wovon hängt die schärfe des Bildes bei der Kamera-Obscura ab?}
Von der Menge an "parallel" verlaufender Lichtstrahlen. Je mehr es sind, umso unschärfer wird das Bild. Darum soll das Loch in der
Lochblende möglichst klein ausfallen.\\

\textbox{
  Theoretisch würde ein unendlichkleines Loch (wo nur ein Lichtstrahl je Einfallswinkel durchkommt) das perfekt scharfe Bild auf der 
  Bildfläche erzeugen. (Auf Kosten der Helligkeit)
}

\subsection{Wovon hängt die Helligkeit des Bildes bei der Kamera-Obscura ab?}
Von der Menge an "parallel" verlaufender Lichtstrahlen. Je mehr es sind, umso heller wird das Bild. Jedoch geht das auf Kosten der 
Schärfe. Instument zur Manipulation ist also das Loch in der Lochblende, mithilfe welcher das einfallende Licht gesteuert werden
kann.\\

\subsection{In welchem Zusammenhang stehen Helligkeit und Schärfe bei der Kamera-Obscura im Bezug auf die Fokuslänge?}
Bei der Camera Obscura besteht ein direkter Zusammenhang zwischen Lochgröße, Helligkeit und Schärfe. Je kleiner das Loch, 
desto schärfer das Bild, aber desto geringer die Lichtstärke (verleichbar mit kurzer Belichtungszeit). Umgekehrt gilt: Je größer die 
Blende, desto heller das Bild, weil mehr Licht in die Kamera trifft.\\

\noindent
Die Fokuslänge bei der Camera Obscura der physische Abstand zwischen der Lochblende und der Projektionsfläche, wobei dieser
Abstand die Bildgröße und die optimale Schärfe beeinflusst.

\includeImgNoUrl{H}{width=\linewidth}{./img/Lineare_Abbildungssysteme/3.png}{Fokuslänge}{fig:linsys-3}{\centering}

\subsection{Was sind intrinsische und extrinische Ebenen, Parameter?}
Intrinsische Kameraparameter beschreiben die internen geometrischen Eigenschaften einer Kamera und ermöglichen die Abbildung zwischen 
Kamerakoordinaten und Pixelkoordinaten im Bildsystem. Wo hingegen extrinsische Kamera Parameter die Position der Kamera
im dreidimensionalen Raum beschreiben.

\subsection{Welche intrinischen, extrinsischen Parameter gibt es?}

\subsection{Welche Arten der Verzerrung gibt es bei der Kamera-Kalibrierung?}

\subsection{Wie werden Fehler bei der Kamera-Kalibrierung minimiert?}

\subsection{Was passiert auf dem Bild auf Folie 37 (Kamera Kalibrierung, x und w auf Linie bringen)?}

\subsection{Wie funktioniert Kamera Kalibrierung? (Allgemein und Übung)}

\subsection{Nennen sie die Matrizen für die intrinischen und extrinsischen Parameter?}

\subsection{Was sind Vorteile der intrinischen im Vergleich zu den extrinischen Parametern? (Kommutativität)}

\subsection{Wie können die Kamera Parameter angewendet/umgerechnet werden?}

\subsection{Wie funktionieren 2D Koordinatensysteme?}

\subsection{Was ist der Einheitsvektor in einem 2D-Koordinatensystem?}

\subsection{Was ist mit "Projektion" gemeint? (2D-Koordinatensystem)}

\subsection{Welche Matrix-Operationen können im 2D-Koordinatensystem angewendet werden? (Beispiele)}


