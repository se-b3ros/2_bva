\definecolor{dkgreen}{rgb}{0,0.6,0}
\definecolor{gray}{rgb}{0.5,0.5,0.5}
\definecolor{mauve}{rgb}{0.58,0,0.82}
\lstset{language=SQL,
  basicstyle={\small\ttfamily},
  belowskip=3mm,
  breakatwhitespace=true,
  breaklines=true,
  classoffset=0,
  columns=flexible,
  commentstyle=\color{dkgreen},
  framexleftmargin=0.25em,
  frameshape={}{yy}{}{}, %To remove to vertical lines on left, set `frameshape={}{}{}{}`
  keywordstyle=\color{blue},
  numbers=left, %If you want line numbers, set `numbers=left`
  numberstyle=\tiny\color{gray},
  showstringspaces=false,
  stringstyle=\color{mauve},
  tabsize=2,
  xleftmargin =1em
}

\definecolor{bluekeywords}{rgb}{0,0,1}
\definecolor{greencomments}{rgb}{0,0.5,0}
\definecolor{redstrings}{rgb}{0.64,0.08,0.08}
\definecolor{xmlcomments}{rgb}{0.5,0.5,0.5}
\definecolor{types}{rgb}{0.17,0.57,0.68}
\lstdefinelanguage{xxx}{
  captionpos=b,
  numbers=left, %Nummerierung
  numberstyle=\tiny\color{gray},
  frame=lines, % Oberhalb und unterhalb des Listings ist eine Linie
  showspaces=false,
  showtabs=false,
  breaklines=true,
  showstringspaces=false,
  breakatwhitespace=true,
  escapeinside={(*@}{@*)},
  commentstyle=\color{greencomments},
  morekeywords={partial, var, value, get, set},
  keywordstyle=\color{bluekeywords},
  stringstyle=\color{redstrings},
  basicstyle=\ttfamily\small,
  morekeywords={abstract, as, base, bool, break, byte, case, catch, char, checked, class, const, continue, decimal, default, delegate, do, double, else, enum, event, explicit, extern, false, finally, fixed, float, for, foreach, goto, if, implicit, in, int, interface, internal, is, lock, long, namespace, new, null, object, operator, out, override, params, private, protected, public, readonly, ref, return, sbyte, sealed, short, sizeof, stackalloc, static, string, struct, switch, this, throw, true, try, typeof, uint, ulong, unchecked, unsafe, using, virtual, void, volatile, while},
  sensitive=true,
  morecomment=[l]{//},
  morecomment=[s]{/*}{*/},
  morestring=[b]",
  frameshape={}{yy}{}{}, %To remove to vertical lines on left, set `frameshape={}{}{}{}`
}

\lstdefinelanguage{plaintext}{
  sensitive=false, % keine Sensitivität (Klein- und Großschreibung wird nicht beachtet)
  morecomment=[l]{//}, % Kommentar für "plain text" - nicht notwendig, aber vorhanden
  morestring=[b]", % erlaubt Zeichenketten, falls nötig
  otherkeywords={} % keine speziellen Keywords
}

% Definition für C++
\lstdefinelanguage{cpp}{
  morekeywords={alignas, alignof, and, and_eq, asm, auto, bitand, bitor, bool, break, case, catch, char, char16_t, char32_t, class, const, constexpr, const_cast, continue, decltype, default, delete, do, double, dynamic_cast, else, enum, explicit, export, extern, false, final, friend, goto, if, inline, int, long, mutable, namespace, new, noexcept, not, not_eq, nullptr, operator, or, or_eq, private, protected, public, reinterpret_cast, return, short, signed, sizeof, static, static_assert, static_cast, struct, switch, template, this, thread_local, throw, true, try, typedef, typeid, typename, union, unsigned, using, virtual, void, volatile, wchar_t, while, xor, xor_eq},
  sensitive=true, % Beachtet Groß- und Kleinschreibung
  morecomment=[l]{//}, % Einzelzeilen-Kommentare
  morecomment=[s]{/*}{*/}, % Mehrzeilige Kommentare
  morestring=[b]", % Zeichenketten, die in Anführungszeichen gesetzt sind
}
